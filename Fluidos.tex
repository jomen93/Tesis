\chapter{Fluidos}

\textbf{Quizás dar la introducción con un poco de historia }.\\

El estudio de los fluidos es de vital importancia, dado que en la actualidad se tienen aplicaciones industriales y tecnológicas con un gran impacto social. Son sustancias con características muy particulares, cuando se utiliza la expresión fluido se refiere a líquidos y gases, además dada su composición molecular se consideran como un medio continuo, esto quiere decir que para el estudio de la dinámica del sistema se consideran volúmenes diferenciales que son pequeños respecto del fluido macroscópico pero grandes en relación con las distancias intermoleculares de la sustancia.

\noindent La descripción matemática de un fluido se hace a través de un campo vectorial de velocidades $v = v(x,y,z,t)$ que se supone una función finita y continua en las variables espaciales, para asegurar esto se exige que las primeras derivadas sean continuas en todo el dominio para verificar que las funciones estén representando correctamente la física del campo continuo. Es importante mencionar que la velocidad descrita del fluido es la que pasa por puntos fijos definidos $(x,y,z)$ en un tiempo $t$ determinado, la descripción se hace con puntos fijos y no con partículas fijas dentro del fluido. Tres campos escalares correspondientes a las magnitudes termodinámicas. La primera de ellas es la presión $p = p(x,y,z,t)$ que caracteriza la fuerza interna (stress) la cual representa la fuerza por unidad de área que por definición es una cantidad macroscópica, Temperatura $T = T(x,y,z,t)$,densidad $\rho = \rho(x,y,z,t)$ y la ecuación de estado, estas cantidades determinan la dinámica del sistema. 




\section{Ecuaciones de Navier-Stokes}

Los fluidos en primera aproximación están descritos por la ecuación de Euler , que en resumen es imponer la conservación de la masa, momento y energía en el fluido\cite{Landau}

\noindent Inicialmente se construye la ecuación de Euler considerando que es un proceso irreversible, dado que no se suponen pérdidas de energía\cite{George}, pero este caso es diferente, se escribe la ecuación de Euler 

\begin{eqnarray}
\frac{\partial (\rho v_{i})}{\partial t} = -\frac{\partial \Pi_{ik}}{\partial x_{k}}
\end{eqnarray}

Donde $\Pi_{ik}$ es el tensor de densidad de flujos de impulso, definido de la siguiente manera.

\begin{eqnarray}
\Pi_{ik} = p\delta_{ik} + \rho v_{i}v_{k}
\end{eqnarray}


\noindent Esta cantidad representa una transferencia de momento totalmente reversible dada por el transporte mecánico de las partículas de fluido en el espacio. El punto de partida para obtener la dinámica de Navier Stokes será considerar que la viscocidad se debe a una transferencia de momento en donde no se conserva la energía, por tanto se tiene un proceso irreversible , de unos lugares donde la velocidad es grande a otros donde la velocidad es pequeña. Entonces en la definición del tensor de impulsos se adiciona un término que da cuenta de la transferencia de impulso viscoso. por lo tanto

\begin{eqnarray}
\Pi_{ik} = p\delta_{ik} + \rho v_{i}v_{k} - \sigma^{'}_{ik}=-\sigma_{ik}+\rho v_{i}v_{k}
\end{eqnarray}

\noindent Definiendo el tensor de tensiones $\sigma_{ik} $ y análogamente el tensor de tensiones de la viscocidad $\sigma^{'}_{ik}$, Expresa la parte de momento que no se transfiere mecánicamente, es decir, solamente por procesos de fricción interna.

\noindent Para encontrar la forma del tensor de tensiones de la viscosidad se estudia el fundamento de las partículas de fluido. La única manera de que exista rozamiento es asumir que entre las partículas de fluido tienen velocidades diferentes, por lo tanto existe un movimiento relativo de las mismas. Para el estudio se hacen dos aproximaciones; en la primera aproximación, se asume que $\sigma^{'}_{ik}$ depende de las derivadas espaciales de la velocidad, además que solo depende de las primeras derivadas y su relación es lineal, puesto que $\sigma^{'}_{ik}$ debe anularse para $v = \text{cte}$. La segunda condición que se impone es que $\sigma^{'}_{ik}$ debe anularse cuando se tiene rotación uniforme, dado que en tal situación no se produce rozamiento interno en el fluido

\noindent El tensor que cumple tales condiciones tiene la forma siguiente

\begin{eqnarray}
\sigma^{'}_{ik} = \eta\left(\frac{\partial v_{i}}{\partial x_{k}} + \frac{\partial v_{k}}{\partial x_{i}} - \frac{2}{3}\delta_{ik}\frac{\partial v_{l}}{\partial x_{l}}\right) + \zeta\delta_{ik}\frac{\partial v_{l}}{\partial x_{l}}
\end{eqnarray}

\noindent  Definiendo los coeficientes viscosos $\eta = \eta(T,\rho,P)$ y $\zeta=\zeta(T,\rho,p)$. Los fluidos que dependen de manera lineal con el strain \cite{Landau} se denominan fluidos ideales. Entonces la ecuación de movimiento que se tiene es

\begin{eqnarray}
\rho\left(\frac{\partial u_{i}}{\partial t} + u_{k}\frac{\partial}{\partial x_{k}}u_{i}\right) = -\frac{\partial p}{\partial x_{i}} + \frac{\partial \sigma^{'}_{ik}}{\partial x_{k}}
\end{eqnarray}

\noindent Por tanto

\begin{eqnarray}
\frac{\partial \sigma^{'}_{ik}}{\partial x_{k}} &=& \eta\left(\frac{\partial^{2} v_{i}}{\partial x_{k}\partial x_{k}}+ \frac{\partial^{2} v_{k}}{\partial x_{k}\partial x_{i}}-\frac{2}{3}\delta_{ik} \frac{\partial^{2} v_{l}}{\partial x_{k} \partial x_{l}}\right)+\zeta \frac{\partial^{2} v_{l}}{\partial x_{k}\partial x_{l}}\nonumber \\
&=&\eta\left(\frac{\partial^{2} v_{i}}{\partial x_{k}\partial x_{k}} + \frac{\partial}{\partial x_{i}}\left(\frac{\partial v_{k}}{\partial x_{k}}\right)-\frac{2}{3}\delta_{ik}\frac{\partial}{\partial x_{k}}\left(\frac{\partial v_{l}}{\partial x_{l}}\right)\right) + \zeta \frac{\partial^{2} v_{l}}{\partial x_{k}\partial x_{l}}\nonumber \\
&=& \eta\left(\frac{\partial^{2} v_{i}}{\partial x_{k}\partial x_{k}}\right) + \frac{1}{3}\eta\frac{\partial}{\partial x_{i}}\left(\frac{\partial v_{l}}{\partial x_{l}}\right)+ \zeta \frac{\partial^{2} v_{l}}{\partial x_{k}\partial x_{l}}\nonumber \\
&=& \eta\frac{\partial^{2} v_{i}}{\partial x_{k}\partial x_{k}} + \left(\zeta + \frac{1}{3}\eta\right)\frac{\partial}{\partial x_{i}}\frac{\partial v_{l}}{\partial x_{l}}
\end{eqnarray}


\noindent La ecuación que describe un fluido en general está dada por 

\begin{eqnarray}
\rho\left(\frac{\partial u_{i}}{\partial t} + u_{k}\frac{\partial}{\partial x_{k}}u_{i}\right) = -\frac{\partial p}{\partial x_{i}} + \eta\frac{\partial^{2} v_{i}}{\partial x_{k}\partial x_{k}} + \left(\zeta + \frac{1}{3}\eta\right)\frac{\partial}{\partial x_{i}}\frac{\partial v_{l}}{\partial x_{l}}
\end{eqnarray}


\noindent Que de forma vectorial se escribe de la siguiente forma

\begin{eqnarray}
\rho \left(\frac{\partial \vec{v}}{\partial t}+(\vec{v} \cdot \nabla )\vec{v}\right) = -\nabla p +\eta \nabla^{2}\vec{v} + \left(\zeta + \frac{1}{3}\eta\right)\nabla(\nabla \cdot \vec{v}) 
\end{eqnarray}

\noindent Cuando el fluido se considera incompresible se asume a priori el hecho de $\nabla \cdot \vec{v} = 0$

\begin{eqnarray}
\label{NS}
\frac{\partial \vec{v}}{\partial t}+(\vec{v} \cdot \nabla )\vec{v} = -\frac{1}{\rho}\nabla p +\frac{\eta}{\rho} \nabla^{2}\vec{v}
\end{eqnarray}

\noindent Que es la ecuación de Navier Stokes en donde el tensor de tensiones en un fluido incompresible está dado por 

\begin{eqnarray}
\sigma_{ik} = -p\delta_{ik} + \eta\left(\frac{\partial v_{i}}{\partial x_{k}} + \frac{\partial v_{k}}{\partial x_{i}  }\right)
\end{eqnarray}

\noindent La viscosidad de un fluido incompresible queda determinada por el coeficiente $\eta$ (viscosidad dinámica).Por último se escriben las ecuaciones de Navier Stokes en términos de ecuaciones de conservación y presentan la siguiente forma 

\begin{eqnarray}
\label{continuidad}
\frac{\partial \rho}{\partial t}+\frac{\partial }{\partial x_{k}}(\rho u_{k})&=& 0 \qquad \\
\label{NS}
\frac{\partial(\rho u_{i})}{\partial t}+\frac{\partial}{\partial x_{k}}(\rho u_{k}u_{i})&=&-\frac{\partial p}{\partial x_{i}}+\frac{\partial \sigma_{ij}}{\partial x_{j}} \qquad 
\end{eqnarray}

\section{Turbulencia}

Cuando se estudian las ecuaciones  Navier Stokes se intenta dar una descripción matemática y modelar el movimiento de los fluidos bajo ciertas simplificaciones, sin embargo en la naturaleza se presentan muchas más interacciones que se deben tener en cuenta a la hora de esta tarea. El objetivo que se persigue será el de encontrar un función matemática que describa completamente un sistema físico , esta solución requiere estabilidad y exactitud para poder considerarla como provechosa y utilizarla en aplicaciones concretas, a partir de este punto se encuentra con el concepto de turbulencia. En nuestro entorno cotidiano se encuentran muchos ejemplos familiares del comportamiento turbulento, la interacción de nubes en el cielo, flujo en tuberías en aplicaciones de ingeniería, mezcla de sustancias en procesos químicos, flujo de aire a través de carros, submarinos y aviones, la fotosfera solar y muchos ejemplos interesantes, el estudio y descripción de la turbulencia se presenta entonces como una actividad interdisciplinaria la cual tiene un rango muy amplio de aplicaciones.  Dado que no se tiene un modelo matemático formal para atacar y encontrar soluciones se utiliza la estadística para afrontar este problema, por tal razón una de las características es la aleatoriedad en tanto nuestros análisis se basan en métodos estadísticos. 

La turbulencia causa un mezclado rápido e incrementa la tasa de transferencia de momento, calor  y masa, de hecho la difusividad es uno de los tópicos más interesantes a la hora de análizar la dinámica del sistema, estos comportamientos aparecen siempre con altos números de Reynolds entonces vemos que las inestabilidades se originan en la interacción de los términos viscosos y la no linenalidad en los términos inerciales en las ecuaciones de movimiento \eqref{NS}, esta interacción es muy compleja dado que las matemáticas de las ecuaciones diferenciales parciales y no lineales no tiene soluciones concretas actualmente, aún no se tienen herramientas poderosas para atacar el problema, de hecho podemos mencionar que es uno de los problemas actuales sin resolver en la física.





