\chapter{Teoría de Boltzmann}


El objetivo de la teoría cinética es describir las propiedades macroscópicas de los gases, tales como: presión, temperatura, conductividad térmica, viscosidad, etc; a partir de variables microscópicas que componen los gases como la velocidad, energía cinética e interacción entre otras. Los fundamentos de la teoría moderna fueron establecidos por Maxwell, quien propuso modelos estadísticos y un concepto de ecuación de transporte para poder modelar el comportamiento macroscópico. Esta teoría obtuvo un nuevo impulso a finales del siglo XIX con el trabajo de Boltzmann, quien propuso una ecuación integro-diferencial (ecuación de Boltzmann) la cual representa la evolución de la función de distribución en el espacio de fase. 

\medskip

\noindent Se consideran, en este trabajo, gases compuestos por una gran cantidad de moléculas las cuales se mueven independientemente unas de otras y colisionan entre sí o con las paredes del contenedor. Se asume también que las energías de interacción entre partículas son despreciables en comparación con sus energías cinéticas.

\section{Espacio de fase}

Para definir el espacio de fase se consideran N partículas en tres dimensiones con coordenadas generalizadas $q_{1}, ..., q_{3N}$ y momentos canónicamente conjugados  $p_{1}, ... , p_{3N}$. Llamamos espacio de fase $\Gamma$, al espacio que abarca $6N$ coordenadas. En este espacio, el estado del sistema a un tiempo dado $t$, es representado por un punto a partir de $6N$ valores, $3N$ debido a las componentes del vector de posición y $3N$ de las velocidades de las $N$ partículas.

\medskip

\noindent Generalmente los sistemas físicos que se estudian en la naturaleza se componen de una gran cantidad de partículas, de esta manera el espacio de fase tiene muchas dimensiones y es muy difícil determinar un punto en $\Gamma$. Para simplificar el estudio de la física a nivel macroscópico es necesario calcular cantidades como la energía, el volumen, etc. Estas se calculan conociendo un gran numero de micro estados, es decir, una colección de muchos puntos en el espacio de fase, sin embargo no será posible encontrar todos los micro estados y se hace necesario caracterizarlos mediante una función de distribución, adoptando una visión probabilística en $\Gamma$. 

\medskip

\noindent Un punto en el espacio de fase se escribe como $(q,p) = (q_{1}, ... ,q_{3N}, p_{1}, ... ,p_{3N})$ y se considera una densidad de probabilidad de ocurrencia en el tiempo t como $f(q,p,t)$. Esta distribución de probabilidad describe propiedades estadísticas de los estados en $\Gamma$ y permite analizar casos microscópicos dentro de resultados macroscópicos. En adelante el desafío será encontrar la función de distribución correspondiente a la situación física  particular. (\textbf{label seccion)}.

\medskip

\noindent Para un gran número de partículas N tenemos \cite{Schwabl}

\begin{eqnarray}
f(q,p,t)dqdp = f(q_{1}, ... , q_{3N},p_{1},..,p_{3N},t)\prod_{i=1}^{3N}dq_{i}dp_{i},
\end{eqnarray}

\noindent que representa la probabilidad de encontrar un sistema en un tiempo $t$, con un volumen en el espacio de fase $dqdp$, en la vecindad del punto $(q,p)$ en el espacio $\Gamma$. $f(q,p,t)$ es llamada la \textbf{función de distribución}, la cual debe ser positiva $f(q,p,t) \geq 0$ y normalizable. 

\begin{eqnarray}
f = f(\vec{r},\vec{v},t) \qquad f:{\rm I\!R}^{3}\times{\rm I\!R}^{3}\times{\rm I\!R}^{+}\longrightarrow{\rm I\!R}^{+}
\end{eqnarray}

%\section{Teorema de Louville}
%Se quiere determinar la dependencia temporal de la función de distribución 
%f(q,p,t), supongamos una distribución inicial $f(q_{o},p_{o})$ en $t = 0$


\section{Función de distribución de Maxwell}

La principal pregunta es, ¿cómo conocer la probabilidad de encontrar una molécula que se mueve en un rango de velocidades $\vec{v}+d\vec{v}$? Para poder responder tal pregunta se consideran tres suposiciones físicas \cite{lecture2}

\begin{itemize}
    \item[$\mathbf{\bullet}$] El número de partículas con velocidad $\vec{v}$ es proporcional a $d\vec{v}$. Entonces se considera que $\phi(\vec{v})d\vec{v}$ representa la probabilidad de encontrar una molécula con velocidad $\vec{v}$ en un rango $\vec{v} + d\vec{v}$
    
    \item[$\mathbf{\bullet}$] La probabilidad de encontrar partículas con velocidad $\vec{v}$ es independiente de los grados de libertad (en cada coordenada)

    \begin{eqnarray}
    \phi(u)du\ \phi(v)dv\ \phi(w)dw,
    \end{eqnarray}
    
    entonces se define una función de distribución que depende de las probabilidades de la siguiente manera
    
    \begin{eqnarray}
    f(u,v,w)dudvdw = n\phi(u)du\phi(v)dv\phi(w)dw,
    \end{eqnarray}
    
    donde $n$ denota la densidad de partículas dentro del diferencial del espacio de fase.
    
    \item[$\mathbf{\bullet}$] \textbf{Hipótesis de isotropía:} En el proceso de equilibrio, en ausencia de fuerzas externas, no se distingue entre las tres direcciones, es decir, se asume que las funciones de distribución tridimensionales se componen del producto de tres funciones unidimensionales
    
    \begin{eqnarray*}
    f(u,v,w) = n\phi(u)\phi(v)\phi(w) =\Phi(|\vec{v}|),
    \end{eqnarray*}
    
    entonces se toma el logaritmo para obtener
    
    \begin{eqnarray*}
    \ln \Phi(|\vec{v}|) = \ln n + \ln\phi(u)+\ln\phi(v)+\ln\phi(w),
    \end{eqnarray*}
    
    derivando para cada componente se obtiene un resultado interesante, que impone una condición de vital importancia
    
    
    \begin{eqnarray}
    \frac{1}{|\vec{v}|}\frac{d\ln\Phi(|\vec{v}|)}{d|\vec{v}|} = \frac{1}{u}\frac{d(\ln\phi(u))}{du}=\frac{1}{v}\frac{d\ln\phi(v)}{dv}=\frac{1}{w}\frac{d\ln\phi(w)}{dw},
    \end{eqnarray}
    
    la única posibilidad para que se cumplan las anteriores relaciones es que cada una sea igual a una constante. Integrando se llega a la siguiente expresión 
    
    \begin{eqnarray}
    \label{Fdistribucion}
    \phi(|\vec{v}|) = \alpha^{3} e^{-\beta v^{2}} \qquad\qquad  f(u,v,w) = n\alpha^{3}e^{-\beta v^{2}}
    \end{eqnarray}
    
    donde $\alpha$ y $\beta$ son constantes de integración postivas. Se elige negativo el exponente para que la función sea normalizable, es decir, sea una función de decreciemiento rápido, ideal para la descripción de cantidades físicas \cite{strichartz}.
    
    Ahora se determinan las constantes de integración en el equilibrio termodinámico, y así las variables de estado que se  utilizan son la densidad $\rho$ y la densidad de energía interna $\rho\epsilon$. 
    
    \begin{eqnarray}
    \label{rho}
    \rho(\vec{r},t) &=& mn = \int_{{\rm I\!R^{3}}}mf(\vec{r},\vec{v},t)d\vec{v}\\
    \label{denener}
    \rho(\vec{r},t)\epsilon &=& \frac{3}{2}nKT=\int_{{\rm I\!R^{3}}}\frac{1}{2}mv^{2}f(\vec{r},\vec{v},t)d\vec{v}
    \end{eqnarray}
    
    la ecuación \eqref{denener} expresa que la energía cinética de traslación promedio  por partícula está dada por el teorema de la equipartición de la energía. Integrando las anteriores expresiones y utilizando \eqref{Fdistribucion} se llega a las siguientes relaciones \footnote{Para calcular las integrales de la teoría cinética se puede utilizar la función Gamma 
    \begin{eqnarray}
    \int_{0}^{\infty}x^{n}e^{-\alpha x^{2}}dx = \frac{1}{2}\Gamma\left(\frac{n+1}{2}\right)\left(\frac{1}{\alpha}\right)^{\frac{n+1}{2}}
    \end{eqnarray}
     }
     
     \begin{eqnarray}
     \label{calculoint}
     \rho = \rho\alpha^{3}\left(\frac{\pi}{\beta}\right)^{3/2} , \qquad \frac{3}{2}nkT =\frac{3}{4}mn\alpha^{3}\left(\frac{\pi}{\beta}\right)^{3/2} \frac{1}{\beta}
     \end{eqnarray}
     
     encontrando,  a partir de las relaciones \eqref{calculoint},
     
     \begin{eqnarray}
     \boxed{
     \alpha = \left(\frac{m}{2\pi kT}\right)^{1/2}, \qquad \beta=\frac{m}{2kT}
     }
     \end{eqnarray}
    
    
    por lo tanto la función de distribución está dado por la siguiente expresión 
    
    \begin{eqnarray}
    f(\vec{r},\vec{v},t) = \rho(\vec{r},t)\left(\frac{m}{2\pi kT}\right)^{3/2}e^{-mv^{2}/2kT}.
    \end{eqnarray}

\end{itemize}

\noindent Esta es la distribución de Maxwell-Boltzmann , que modela la evolución en el tiempo de cada partícula en una posición y velocidad en particular. Sí se conoce $f$, las cantidades macroscópicas observables pueden calcularse a partir de funciones microscópicas. En particular se tienen los momentos macroscópicos  \cite{lecture1}

\begin{eqnarray}
\rho(\vec{r},t)&=&mn(\vec{r},t)=m\int_{{\rm I\!R}^{3}}f(\vec{r},\vec{v},t)d\vec{v}\\
\vec{u}(\vec{r},t) &=& \frac{1}{n(\vec{v},t)}\int_{{\rm I\!R}^{3}}\vec{v} f(\vec{r},\vec{v},t)d\vec{v}
\end{eqnarray}

\noindent que son respectivamente la densidad de masa y la densidad de velocidad. Ahora definimos la energía traslacional de la forma

\begin{eqnarray}
\epsilon(\vec{r},t)=\frac{1}{n(\vec{r},t)}\int_{{\rm I\!R}^{3}}(\vec{v}-\vec{u})^{2}f(\vec{r},\vec{v},t)d\vec{v}
\end{eqnarray}




\section{Ecuación de Boltzmann}



\noindent La función de distribución $f$ depende  de $\vec{x}$, $\vec{\xi}$ y $t$. Veamos su variación temporal

\begin{eqnarray}
\frac{df}{dt} = \left(\frac{\partial f}{\partial x_{\alpha}}\right)\frac{dx_{\alpha}}{dt}+\left(\frac{\partial f}{\partial \xi_{\alpha}}\right)\frac{d\xi_{\alpha}}{dt}+\left(\frac{\partial f}{\partial t}\right)
\end{eqnarray}

\noindent En donde $dx_{\alpha}/dt$ es la velocidad de las partículas a nivel mesoscópico $\xi_{\alpha}$. Por otro lado, el segundo término está relacionado con la aceleración y así, teniendo en cuenta la segunda ley de Newton, podemos escribir la densidad de fuerza como $d\xi_{\alpha}/dt = F_{\alpha}/\rho$. Por lo tanto la ecuación de evolución que tenemos está dada por 

\begin{eqnarray}
\frac{df}{dt} = \xi_{\alpha}\left(\frac{\partial f}{\partial x_{\alpha}}\right)+\frac{F_{\alpha}}{\rho}\left(\frac{\partial f}{\partial \xi_{\alpha}}\right)+\frac{\partial f}{\partial t}
\end{eqnarray}

\noindent en general, el lado izquierdo de la ecuación es un término fuente que indica la tasa de cambio de la función de distribución dada por las colisiones entre partículas.  La siguiente ecuación es llamada \emph{Ecuación de Boltzmann}

\begin{eqnarray}
\label{Boltzmann}
\frac{\partial f}{\partial t} + \xi_{\alpha}\left(\frac{\partial f}{\partial x_{\alpha}}\right)+\frac{F_{\alpha}}{\rho}\left(\frac{\partial f}{\partial \xi_{\alpha}}\right)=\Omega(f)
\end{eqnarray}

\noindent donde se ha definido el operador de colisión al termino $\Omega(f)$ el cual puede tener diversas formas, existe un amplio campo de investigación al respecto \cite{lecture3}\cite{lecture4}\cite{lecture5}. Sin embargo debemos exigir a tal operador que conserve tres cantidades físicas importantes, la masa, el momento y la energía. Matemáticamente se expresan como sigue 

\begin{eqnarray}
\int\Omega(f)d\vec{\xi} = 0 \qquad \int\vec{\xi}\Omega(f)d\vec{\xi} = 0 \qquad \int\xi^{2}\Omega(f)d\vec{\xi} = 0.
\end{eqnarray}

\noindent para deducir el operador de colisión se necesita resolver dos  integrales complicadas sobre todo el espacio \cite{lecture5}, que en esencia es considerar todas las posibles formas de colisión que puedan tener las partículas en estudio para una fuerza en particular. Un operador que cumple con tales condiciones fue el propuesto por Bhatnagar, Gross y Krook en 1954 \cite{BGK}, entonces se asume que 

\begin{eqnarray}
\Omega(f) = -\frac{1}{\tau}(f-f^{(0)})
\end{eqnarray}

\noindent en la anterior ecuación $\tau$ es el tiempo de relajación, el cual cuantifica que tan rápido chocan las partículas 

\section{Ecuaciones de conservación macroscópicas}

\noindent Tomando los momentos de la ecuación de Boltzmann \eqref{Boltzmann} se pueden encontrar las ecuaciones de conservación del sistema tales como la masa, el momento y la energía. Se define entonces

\begin{eqnarray}
\label{momentos}
\Pi_{o} = \int f d\vec{\xi} = \rho &\qquad& \Pi_{\alpha} = \int \xi_{\alpha}f d\vec{\xi} = \rho u_{\alpha}\\
\Pi_{\alpha\beta} = \int \xi_{\alpha\beta}f d\vec{\xi}  &\qquad& \Pi_{\alpha\beta\gamma} = \int \xi_{\alpha}\xi_{\beta}\xi_{\gamma}f d\vec{\xi} \nonumber
\end{eqnarray}

\noindent Para la conservación de la masa se integra \eqref{Boltzmann} obteniendo. 

\begin{eqnarray}
\frac{\partial}{\partial t}\int f d\vec{\xi}+\frac{\partial}{\partial x_{\alpha}}\int \xi_{\alpha}fd\vec{\xi}+\frac{F_{\alpha}}{\rho}\int \frac{\partial f}{\partial \xi_{\alpha}}d\vec{\xi} = \int \Omega(f)d\vec{\xi}
\end{eqnarray}

\noindent Notando que $t$ y $x$ no son funciones de $\xi$, podemos permutar la integral con la divergencia y reorganizar términos. Falta hacer el calculo del término de fuerza utilizando la integración por partes multidimensional\footnote{Ecuacion de integración por partes multidimensional \begin{eqnarray}
\label{partes}
\int_{V} \frac{\partial u}{\partial x_{\alpha}} = \int_{S}uv dS_{\alpha} - \int_{V}u\frac{\partial v}{\partial x_{\alpha}}\nonumber
\end{eqnarray}
} y asumiendo que las integrales de superficie se desvanecen cuando $\vec{\xi}\rightarrow  0$. Se impone esta última condición para asegurar que las ecuaciones de conservación sean finitas. Entonces se llega a la siguiente ecuación utilizando \eqref{momentos}


\begin{eqnarray}
\boxed{\frac{\partial \rho}{\partial t} +\frac{\partial \rho u_{\alpha}}{\partial x_{\alpha}}= 0 }
\end{eqnarray}

\noindent que es la ecuación de continuidad \cite{Landau}. La importancia de este resultado radica en que la forma de la ecuación de continuidad se mantendrá sin importar la forma funcional especifica de la función distribución, de tal manera que la definición de los momentos en la ecuación tienen mayor relevancia debido a que conectan el mundo mesoscópico con las ecuaciones macroscópicas.

\medskip

\noindent Para el caso de la siguiente ecuación de conservación se toma el primer momento de \eqref{Boltzmann} y se hace la integral en todo el espacio de velocidades, obteniendo la siguiente relación 



\begin{eqnarray}
\frac{\partial}{\partial t}\int \xi_{\alpha}f d\vec{\xi}+\frac{\partial}{\partial x_{\beta}}\int \xi_{\alpha}\xi_{\beta}fd\vec{\xi}+\frac{F_{\beta}}{\rho}\int \xi_{\alpha}\frac{\partial f}{\partial \xi_{\beta}}d\vec{\xi} = \int \xi_{\alpha}\Omega(f)d\vec{\xi}
\end{eqnarray}

\noindent Se utiliza la integración por partes para encontrar el momento relacionado con el término de la fuerza, encontrando que se puede hacer 

\begin{eqnarray}
\int \xi_{\alpha}\frac{\partial f}{\partial \xi_{\beta}}d\vec{\xi} = -\int \frac{\partial \xi_{\alpha}}{\partial \xi_{\beta}}fd\vec{\xi}= -\rho\delta_{\alpha\beta}
\end{eqnarray}

\noindent como resultado podemos ver la ecuación de conservación del momento dada por la siguiente expresión 

\begin{eqnarray}
\frac{\partial (\rho u_{\alpha})}{\partial t}+\frac{\partial \Pi_{\alpha\beta}}{\partial x_{\beta}} = F_{\alpha}
\end{eqnarray}

\noindent dada la definición de $\Pi_{\alpha\beta}$ dada en \eqref{momentos} podemos entender físicamente como el flujo de cantidad de movimiento dirección $\alpha$ de la dirección $\beta$ de momento, en donde se evidencia que es simétrica respecto de índices. Veamos con más detalle $\Pi_{\alpha\beta}$ , definamos $\xi_{\alpha}\xi_{\beta}=(u_{\alpha}+v_{\alpha})(u_{\beta}+v_{\beta})$, entonces obtenemos lo siguiente 

\begin{eqnarray}
\Pi_{\alpha\beta} = \int (u_{\alpha}u_{\beta}+u_{\alpha}v_{\beta}+v_{\alpha}u_{\beta}+v_{\alpha}v_{\beta})fd\vec{\xi} = \rho u_{\alpha}u_{\beta} -\sigma_{\alpha}
\end{eqnarray}

\noindent tomando la condición de simetría de el primer momento $\Pi_{\alpha\beta}=\Pi_{\beta\alpha}$ se encuentra  que $u_{\alpha}v_{\beta} + v_{\alpha}u_{\beta}=0$, esto hace que los términos intermedios de la anterior integral se cancelen, obteniendo los dos términos de la do derecho bajo esta definición de $\sigma$

\begin{eqnarray}
\sigma_{\alpha\beta} = - \int v_{\alpha}v_{\beta}fd\vec{\xi}
\end{eqnarray}

\noindent el cual representa la difusión del momento y está determinado por la forma que se eliga de f, finalmente se tiene

\begin{eqnarray}
\boxed{\frac{\partial (\rho u_{\alpha})}{\partial t}+\frac{\partial(\rho u_{\alpha}u_{\beta})}{\partial x_{\beta}}=\frac{\partial \sigma_{\alpha\beta}}{\partial x_{\beta}}+F_{\alpha}}
\end{eqnarray}

\noindent como última consideración encontramos la ecuación de conservación de energía , tomando el segundo momento de la función de distribución de la ecuación de Boltzmann \eqref{Boltzmann}

\begin{eqnarray}
\frac{\partial}{\partial t}\int \xi_{\alpha}\xi_{\beta}f d\vec{\xi}+\frac{\partial}{\partial x_{\alpha}}\int \xi_{\alpha}\xi_{\beta}\xi_{\gamma}fd\vec{\xi}+\frac{F_{\alpha}}{\rho}\int \xi_{\alpha}\xi_{\beta}\frac{\partial f}{\partial \xi_{\alpha}}d\vec{\xi} = \int \xi_{\alpha}\xi_{\beta}\Omega(f)d\vec{\xi}
\end{eqnarray}

\noindent que por las definiciones de momentos \eqref{momentos} encontramos una ecuación de conservación del siguiente estilo 


\begin{eqnarray}
\frac{\partial (\rho E)}{\partial t} + \frac{1}{2}\frac{\partial \Pi_{\alpha\beta\gamma}}{\partial x_{\alpha}}=F_{\alpha}u_{\alpha}
\end{eqnarray}

En donde $\Pi_{\alpha\beta\gamma}$ representa el flujo de energía en la dirección $\alpha$. Tomando en cuenta el desarrollo en la deducción de el primer momento, calculamos 

\begin{eqnarray}
\frac{1}{2}\Pi_{\alpha\beta\gamma}&=&\frac{1}{2}\int (u_{\alpha}u_{\beta}u_{\gamma}+u_{\alpha}u_{\beta}u_{\gamma}+u_{\alpha}u_{\beta}u_{\gamma}+u_{\alpha}u_{\beta}u_{\gamma})fd\vec{\xi}\nonumber\\
&=&\frac{1}{2}\rho u_{\alpha} u^{2}+\rho u_{\alpha}e- u_{\beta}\sigma_{\alpha\beta}+q_{\alpha}\nonumber\\
&=&\rho u_{\alpha}E - u_{\beta}\sigma_{\alpha\beta} +q_{\alpha}
\end{eqnarray}

\noindent se tiene ahora que el primer término representa la advección macroscópica de la energía, el segundo término como el trabajo hecho por el tensor de "Strain" y se define el último termino como la difusión de la energía de la siguiente manera

\begin{eqnarray}
q_{\alpha} = \frac{1}{2}\int v_{\alpha} v^{2}f d\vec{\xi}
\end{eqnarray}

Finalmente se llega a la ecuación de conservación relacionada con la energía

\begin{eqnarray}
\boxed{
\frac{\partial (\rho e)}{\partial t} + \frac{\partial (\rho u_{\alpha}e)}{\partial x_{\alpha}}= \sigma_{\alpha\beta}\frac{\partial (\sigma_{\alpha\beta}u_{\beta})}{\partial x_{\alpha}}+ F_{\alpha}u_{\alpha}-\frac{\partial q_{\alpha}}{\partial x_{\alpha}}
}
\end{eqnarray}

\newpage









