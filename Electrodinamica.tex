\setcounter{page}{1}
\chapter{Electrodinámica}

\noindent En la presente sección se revisa la teoría de la  electrodinámica clásica. El estudio de la variación temporal de los campos magnéticos y eléctricos muestra una fuerte relación entre ellos, en donde cambios en los campos eléctricos genera campos magnéticos y viceversa.%\cite{Chaichian}\\

\section{Ley de Gauss}

\noindent Esta es la primera de las ecuaciones básicas para la descripción del campo eléctrico, en donde se hace la deducción de la forma matemática para el campo eléctrico generado a partir de una distribución de cargas en reposo. El campo eléctrico dado por una distribución de cargas está dado por la siguiente expresión \cite{Jackson}

\begin{eqnarray}
    \vec{E}(\vec{r}) =\frac{1}{4\pi\epsilon_{0}} \int_{\bf{V}}\rho(\vec{\bf{r}}
)\frac{\vec{r}-\vec{\bf{r}}}{|\vec{r}-\vec{\bf{r}}|^{3}}d\bf{V}
\end{eqnarray}

\noindent se calcula $\nabla \cdot \vec{E}$, es decir,

\begin{eqnarray}
\label{integralGauss}
    \nabla \cdot \vec{E} = \nabla\cdot\left[\int_{V}\rho(\vec{\bf{r}}
)\frac{\vec{r}-\vec{\bf{r}}}{|\vec{r}-\vec{\bf{r}}|^{3}}d\bf{V}\right] = \frac{1}{4\pi\epsilon_{0}}\int_{\bf{V}}\rho(\vec{\bf{r}})\nabla\cdot\left[\frac{\vec{r}-\vec{\bf{r}}}{|\vec{r}-\vec{\bf{r}}|^{3}}\right]d\bf{V}
\end{eqnarray}

\noindent consideramos $\vec{R} = \vec{r}-\vec{\bf{r}}$ , entonces se necesita calcular 

\begin{eqnarray}
    \nabla\cdot\left[\frac{\vec{R}}{R^{3}}\right] &=& \frac{1}{R^{3}}\nabla\cdot\vec{R}+\vec{R}\cdot\nabla\left[\frac{1}{R^{3}}\right]\\
    &=&\frac{3}{R^{3}} - \frac{3}{R^{3}} 
\end{eqnarray}

\noindent cuando se considera $\vec{R} = 0$ se puede ver que el sistema de referencia está ubicado en el mismo lugar que las cargas generadoras del campo eléctrico esto supone un problema en el cálculo por tanto se considera $R\neq0$. Para resolver \eqref{integralGauss} se utiliza una pequeña esfera de radio R alrededor del punto $\vec{r}=\vec{\bf{r}}$, entonces usando el teorema de la divergencia \cite{Arfken}

\begin{eqnarray}
    \int_{\bf{V}}\nabla\cdot\left[\frac{\vec{R}}{R^{3}}\right]dV &=& \oint_{S}\left[\frac{\vec{R}}{R^{3}}\right]\cdot\hat{n}dA = \oint_{S}\left[\frac{\vec{R}}{R^{3}}\right]\cdot\left[\frac{\vec{R}}{R}\right]dA = \oint_{S}\frac{R^{4}}{R^{4}}dA \\
    &=&\frac{1}{R^{2}}\oint_{S}dA = \frac{1}{R^{2}}(4\pi R^{2}) = 4\pi
\end{eqnarray}

\noindent el anterior resultado no depende de $R$ el resultado es válido para $\vec{R} = 0$, que es el punto de interés. Como el resultad se anula para todo punto $\vec{R} \neq 0$. La única contribución se da cuando $\vec{R} = 0$. De esta manera se utilizan las propiedades de la delta de Dirac \cite{Arfken} para escribir 

\begin{eqnarray}
    \int_{\bf{V}}\nabla\cdot\left[\frac{\vec{R}}{R^{3}}\right]dV = \int_{\bf{V}}4\pi \delta(\vec{R})dV
\end{eqnarray}

\noindent reemplzando en \eqref{integralGauss}, se llega a lo siguiente

\begin{eqnarray}
    \label{Leygauss}
    \nabla \cdot \vec{E} = \frac{1}{\epsilon_{0}}\int_{\bf{V}}\rho(\vec{\bf{r}})\delta(\vec{r}-\vec{\bf{r}})d\bf{V} \longrightarrow \boxed{ \nabla \cdot \vec{E} = \frac{\rho}{\epsilon_{0}}}
\end{eqnarray}

\noindent la ley de Gauss determina que la fuente de campo eléctrico está dado por la manera en que se distribuyen las cargas en el espacio.


\section{Ecuación de Gauss para el magnetismo }


\noindent Por analogía con el campo eléctrico se puede plantear la ley de Gauss para el Magnetismo, que tendría la siguiente forma

\begin{eqnarray}
    \oint \vec{B}\cdot\hat{n}\  dA = \mu_{0}Q_{mag}
\end{eqnarray}

\noindent donde se plantea la existencia de la carga magnética, sin embargo hasta la fecha no se ha encontrado la existencia de los monopolos magnéticos, es decir, cargas magnéticas aisladas. Lo único que se tiene comprobado son los dipolos magnéticos, así que la integral propuesta anteriormente se reduce de la siguiente manera

\begin{eqnarray}
    \oint \vec{B}\cdot\hat{n}\  dA = 0
\end{eqnarray}

\noindent esto se cumple para cualquier campo magnético. Para poder demostrar como valida  la anterior expresión se considera la ley de Biot Savart \cite{kleber}

\begin{eqnarray}
    \nabla\cdot\vec{B} = \nabla\cdot\left[\frac{\mu_{o}}{4\pi}\int_{V}\frac{\vec{J}(\vec{\bf{r}})\times(\vec{r}-\vec{\bf{r}})}{|\vec{r}-\vec{\bf{r}}|^{3}}d\bf{V}\right] = \frac{\mu_{o}}{4\pi}\int_{V}\nabla\cdot\left[\frac{\vec{J}(\vec{\bf{r}})\times(\vec{r}-\vec{\bf{r}})}{|\vec{r}-\vec{\bf{r}}|^{3}}\right]d\bf{V}
\end{eqnarray}

\noindent utilizando una identidad  vectorial\footnote{$\nabla\cdot(\vec{A}\times \vec{B}) = (\nabla \times \vec{A})\cdot \vec{B} - \vec{A}\cdot(\nabla \times \vec{B})$} se puede reescribir la divergencia de las corrientes generadoras de campo magnético 

\begin{eqnarray}
    \nabla\cdot\left[\frac{\vec{J}(\vec{\bf{r}})\times(\vec{r}-\vec{\bf{r}})}{|\vec{r}-\vec{\bf{r}}|^{3}}\right] = [\nabla \times \vec{J}(\vec{\bf{r}})]\cdot \frac{\vec{r}-\vec{\bf{r}}}{|\vec{r}-\vec{\bf{r}}|^{3}}- \vec{J}(\vec{\bf{r}})\cdot\left[\nabla\times\frac{\vec{r}-\vec{\bf{r}}}{|\vec{r}-\vec{\bf{r}}|^{3}}\right]
\end{eqnarray}

\noindent dado que $\vec{J}(\vec{\bf{r}})$ no depende de $\vec{r}$ y el operador actúa sobre las coordenadas descritas por $\vec{r}$, se anula el primer término , por tanto se tiene que 

\begin{eqnarray}
     \nabla\cdot\left[\frac{\vec{J}(\vec{\bf{r}})\times(\vec{r}-\vec{\bf{r}})}{|\vec{r}-\vec{\bf{r}}|^{3}}\right] = \vec{J}(\vec{\bf{r}})\cdot\left(\nabla\times\nabla\left[\frac{1}{|\vec{r}-\vec{\bf{r}}|}\right]\right)
\end{eqnarray}

\noindent como el rotacional del gradiente de una función escalar siempre es cero \cite{Arfken}, se llega a 

\begin{eqnarray}
    \nabla\cdot\left[\frac{\vec{J}(\vec{\bf{r}})\times(\vec{r}-\vec{\bf{r}})}{|\vec{r}-\vec{\bf{r}}|^{3}}\right] = 0
\end{eqnarray}

\noindent finalmente se obtiene la primera ecuación para el campo magnético 

\begin{eqnarray}
    \label{divceroB}
    \boxed{\nabla \cdot \vec{B} = 0}
\end{eqnarray}

\noindent Los principios fundamentales de la teoría electromagnética se consigan en un sistema de ecuaciones diferenciales parciales lineales, conocidas como \emph{Ecuaciones de Maxwell}. Dos de tales ecuaciones muestran la variacion temporal de los campos eléctricos $\vec{E}$ y magnéticos $\vec{B}$. Para empezar se consideran estas dos primeras ecuaciones.\\



\section{Ecuación de Ampère-Maxwell}

\noindent Se considera la ecuación de Ampère \cite{Jackson} , aplicando el operador divergencia se obtiene una relación para poder obtener un régimen estacionario



\begin{eqnarray}
    \label{Ampère}
    \nabla \times \vec{H} = \vec{J} \qquad  \longrightarrow  \qquad 
    \nabla \cdot \vec{J} = 0 
\end{eqnarray}

\noindent tal condición junto con la ecuación de continuidad exige al sistema la conservación de la densidad de carga, y esto evidencia una posible contradición de la formulación

\begin{eqnarray}
    \nabla \cdot \vec{J} = -\frac{\partial \rho}{\partial t}
\end{eqnarray}

\noindent Maxwell resuelve tal inconsistencia introduciendo el concepto de corriente de desplazamiento. La densidad de corriente $\vec{J}$ en la ecuación de continuidad aparece debido al desplazamiento de cargas y puede contener corrientes por conducción o conveccion. Pero en el caso de Ampère la corriente está sujeta a la condición dada por  \eqref{Ampère} , falta considerar un tipo de corriente para que sea consistente el modelo, esto es

\begin{eqnarray}
    \vec{J} = \vec{J}_{cond} + \vec{J}_{Des}
\end{eqnarray}

\noindent por lo tanto la ecuación de Ampère toma la forma

\begin{eqnarray}
    \nabla \times \vec{H} = \vec{J}_{cond} + \vec{J}_{Des}
\end{eqnarray}

\noindent tomando la divergencia se puede obtener la siguiente condición 

\begin{eqnarray}
    \nabla \cdot \vec{J}_{Des} = -\nabla \cdot \vec{J}_{cond} = \frac{\partial \rho}{\partial t}
\end{eqnarray}

\noindent considerando el teorema de Gauss \cite{Arfken} se puede llegar a las siguientes expresiones

\begin{eqnarray}
    \nabla\cdot\vec{J}_{Des} =  \nabla\cdot\left(\frac{\partial\vec{D}}{\partial t}\right) \qquad \longrightarrow \qquad \vec{J}_{Des} = \frac{\partial \vec{D}}{\partial t}
\end{eqnarray}


\noindent llegando a la forma reportada en la literatura como la ecuación de Ampère-Maxwell. Considerando también las ecuaciones en medios materiales se obtiene las siguientes expresiones. Teniendo en cuenta las definiciones $H = B/\mu_{o}$, $D = \epsilon_{o}\vec{E}$ 

\begin{eqnarray}
\label{Ampere}
    \boxed{\nabla \times \vec{H} = \vec{J} + \frac{\partial D}{\partial t}} \qquad \qquad \boxed{\nabla \times \vec{B} = \mu_{o}\vec{J} + \frac{1}{c^{2}}\frac{\partial \vec{E}}{\partial t}}
\end{eqnarray}

\noindent donde $c =1/\sqrt{\epsilon_{o}\mu_{o}}$. La corriente de desplazamiento también se da en el vacío dado que la variación temporal del campo eléctrico produce un campo magnético variable 

\section{Ecuación de Faraday-Maxwell}

\noindent La fuerza electromotriz inducida alrededor del circuito es proporcional a la tasa de cambio del flujo magnético que atraviesa la superficie encerrada por el circuito.  \cite{Lacava}

\begin{eqnarray}
\label{Faraday}
    f = \oint_{C} \vec{E}\cdot d\vec{l} = -\frac{d }{dt}\int_{S}\vec{B}\cdot d\vec{S} = -\frac{ d\Phi}{dt}
\end{eqnarray}

\noindent donde $f$ es la fuerza electromotriz ($fem$), $S$ es una superficie rodeada de un contorno que cumple el teorema de Jordan \cite{Rudin} independiente del tiempo, $\Phi$ es el flujo magnético a través de la superficie. El signo en la ecuación \eqref{Faraday} está dado por la ley de Lenz \cite{Purcell}, la cual asegura que  la corriente inducida está en una dirección tal que se opone al cambio de flujo del campo  a través del circuito .\\

\medskip

\noindent Se supone que los sucesos físicos deben ser invariantes, en el sentido marcos de referencia relativos, esto sugiere que dos observadores con diferentes velocidades relativas pero constantes en el tiempo $v$ entre ellos deben percibir fenómenos iguales. En particular en la ley de Faraday se consigue esto considerando un circuito que se mueve a cierta velocidad $v$, por tanto se observa que 

\begin{eqnarray}
\label{Faraday2}
    \oint_{C} \vec{E'}\cdot d\vec{l} = -\frac{d }{dt}\int_{S}\vec{B}\cdot d\vec{S}
\end{eqnarray}

\noindent como la fuerza electromotriz es igual a la derivada total del flujo, se debe considerar que el circuito se puede mover de manera arbitraria, entonces la translación del circuito cambia el lugar de la frontera de la superficie y el flujo cambia en relación con la velocidad $v$ ,por lo tanto, se considera la derivada convectiva dado que se mueve todo el sistema de referencia

\begin{eqnarray}
    \frac{d }{dt}\int_{S}\vec{B}\cdot d\vec{S} &=& \left(\frac{\partial}{\partial t} + \vec{v}\cdot\nabla\right)\int_{S}\vec{B}\cdot d\vec{S}\nonumber\\
    &=& \int_{S}\left(\frac{\partial \vec{B}}{\partial t} + (\vec{v}\cdot\nabla)\vec{B}\right)d\vec{S}\nonumber \\
    &=&\int_{S}\left(\frac{\partial \vec{B}}{\partial t} + \nabla \times (\vec{B} \times \vec{v}) +\vec{v}(\cancel{\nabla \cdot \vec{B}})\right)d\vec{S}\nonumber \\
    &=&\int_{S}\left(\frac{\partial \vec{B}}{\partial t} + \oint_{C}(\vec{B}\times \vec{v})\cdot d\vec{l}\right)
\end{eqnarray}

\noindent entonces se obtiene la ecuación de Faraday \eqref{Faraday2} cuando se considera el circuito C en movimiento 

\begin{eqnarray}
    \oint_{C} \left(\vec{E'} + (\vec{v}\times \vec{B})\right)\cdot d\vec{l} = -\int_{S}\frac{\partial \vec{B}}{\partial t}\cdot d\vec{S}
\end{eqnarray}

\noindent la consideración de la invariancia supone que $\vec{E'} = \vec{E} -( \vec{v}\times \vec{B})$, donde $\vec{E'}$ es el campo eléctrico en reposo, como la consideración es clásica, solo es válida cuando $v << c$. Utilizando el teorema de Stokes \cite{Arfken} en \eqref{Faraday2},se muestra que 

\begin{eqnarray}
\label{FM}
    \int_{S}\left(\nabla \times \vec{E}+\frac{\partial\vec{B}}{\partial t}\right)\cdot d\vec{S} =0 \qquad \longrightarrow \qquad \boxed{\nabla \times \vec{E} + \frac{\partial \vec{B}}{\partial t} = 0}
\end{eqnarray}

\noindent esta es la ecuación de Faraday-Maxwell. Expresa el hecho de que la variación temporal del campo magnético trae como consecuencia un cambio en el espacio del campo eléctrico. En el vacío o en medios materiales.

