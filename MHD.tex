\chapter{Magnetohidrodinámica (MHD)}


\noindent El objetivo concreto de este capítulo es dar un entendimiento físico del modelo ideal magnetohidrodinámico (MHD), esto incluye una descripción básica del modelo.
\begin{comment}

\noindent El modelo ideal MHD se sitúa bajo la consideración de longitud de onda larga y bajas frecuencias. En este orden de ideas y pretendiendo abarcar la física desde una perspectiva más formal, en la deducción del caso ideal se consideran principios básicos. Utilizando las ecuaciones de Maxwell\eqref{Leygauss},\eqref{divceroB},\eqref{Ampere},\eqref{FM} y la ecuación de Boltzmann\eqref{Boltzmann},



\begin{eqnarray}
\nabla \cdot \vec{E} &=& \frac{\rho}{\epsilon_{0}}\nonumber\\
\nabla \cdot \vec{B} &=& 0\nonumber\\
\nabla\times\vec{B} &=& \mu_{o}\vec{J}+\frac{1}{c^{2}}\frac{\partial \vec{E}}{\partial t}\nonumber\\
\nabla\times\vec{E} &=& \frac{\partial \vec{B}}{\partial t}\nonumber\\
\label{BMHD}
  \left(\frac{\partial f}{\partial t}\right)_{c} &=&\frac{\partial f}{\partial t} + v_{i}\frac{\partial f}{\partial x_{i}} + \frac{q(E_{i}+\varepsilon_{ijk}v_{j}B_{k})}{m}\frac{\partial f}{\partial v_{i}}
\end{eqnarray}

\noindent en particular para describir los fenomenos MHD se incluye el termino de fuerza de Lorentz, además las definiciones de densidad de corriente y densidad de carga 

\begin{eqnarray}
\vec{J} &=& q\int \vec{\xi}f(\vec{x},\vec{\xi},t)d\vec{\xi}\\
\sigma &=& q\int f(\vec{x},\vec{\xi},t)d\vec{\xi} d\vec{\xi}
\end{eqnarray}

\noindent donde $f$ es la función de distribución definida en la sección anterior. En la descripción de Boltzmann se consideran dos tipos de fuerzas que actúan sobre las partículas. La primera de ellas es de largo alcanze y está dada por las fuerzas de Lorentz $q(\vec{E}+\vec{v}\times\vec{B})$ en la cual $\vec{E}$ y $\vec{B}$ son de buen comportamiento en el sentido del calculo diferencial y obtenidos a partir de el vector densidad de corriente y la densidad de carga. La segunda fuerza está asociadada con el operador de colisión\cite{BGK} impuesto en la ecuación de Boltzmann, el cual es una interacción de corto alcance. Es bueno recordar que nuestro parámetro de comparación será el camino libre medio\cite{lecture5}. 

\medskip

\end{comment}
Asumimos que el fluido no es relativista y el interés se centra en variaciones de tiempo muy lentas, ado esto podemos considerar ignorar la corriente de desplazamiento al tener en cuenta estas consideraciones. Por lo ranto la ley de Ohm\cite{Jackson}:

\begin{eqnarray}
\label{OHMr}
\vec{E}' = \eta\vec{J}
\end{eqnarray}

\noindent donde $\vec{E}'$ es un campo eléctrico visto desde un sistema de referencia con velocidad relativa $\vec{v}$, por lo tanto el campo eléctrico en tal sistema está dado por 

\begin{eqnarray}
\vec{E}' = \frac{\vec{E}+\vec{v}\times\vec{B}}{\sqrt{1-v^{2}/c^{2}}}
\end{eqnarray}

\noindent como el problema no tiene en cuenta efectos de la relatividad especial se considera una expansión de Taylor y se restringe a primer orden la serie

\begin{eqnarray}
\vec{E}' = \vec{E}+\vec{v}\times\vec{B} + \mathcal{O}\left(\frac{v^{2}}{c^{2}}\right)
\end{eqnarray}

\noindent entonces teniendo en cuenta \eqref{OHMr} y \eqref{Ampère} se llega :

\begin{eqnarray}
\vec{E} = -\vec{v}\times\vec{B}+\eta\vec{J} \qquad \qquad \vec{J} = \frac{1}{\mu_{o}}(\nabla \times \vec{B})
\end{eqnarray}

\noindent se puede utilizar la ecuación de Faraday-Maxwell \eqref{FM}, obteniendo:

\begin{eqnarray}
\label{Btemp}
\frac{\partial \vec{B}}{\partial t} &=& -\nabla \times \vec{E} = \nabla \times(-\vec{v}\times\vec{B}+\eta\vec{J)}\\
&=& -\nabla \times (\eta\vec{J})+\nabla\times(\vec{v}\times\vec{B})\\
&=&-\nabla\times\left(\frac{\eta}{\mu_{o}}(\nabla\times\vec{B})\right) + \nabla\times(\vec{v}\times\vec{B})\\
&=&\frac{\eta}{\mu_{o}}\nabla^{2}\vec{B} + \nabla\times(\vec{v}\times\vec{B})
\end{eqnarray}

\noindent con este procedimiento se elimina la dependencia explícita en las ecuaciones del campo eléctrico, la evolución temporal del campo magnético depende de variaciones espaciales de de $\vec{B}$ y del campo de velocidades del sistema. Para hacer un análisis físico más simple se escribe \eqref{Btemp} en forma conservativa, para tal fin se considera el segundo termino de la anterior ecuación

\begin{eqnarray}
\nabla\times(\vec{v}\times\vec{B}) &=& \varepsilon_{ijk}\frac{\partial}{\partial x_{j}}(\varepsilon_{kml}v_{m}B_{l})\\
&=&\varepsilon_{ijk}\varepsilon_{kml}\frac{\partial}{\partial x_{j}}(v_{m}B_{l})\\
&=&-\varepsilon_{ijk}\varepsilon_{lmk}\frac{\partial}{\partial x_{j}}(v_{m}B_{l})\\
&=&(\delta_{il}\delta_{jm}-\delta_{im}\delta_{jl})\frac{\partial}{\partial x_{j}}(v_{m}B_{l})\\
&=&\frac{\partial}{\partial x_{j}}(v_{i}B_{j})-\frac{\partial}{\partial x_{j}}(v_{j}B_{i})\\
&=&\frac{\partial}{\partial x_{j}}(v_{i}B_{j}-v_{j}B_{i})
\end{eqnarray}

\noindent por tanto para el campo magnético  la ecuación de conservación es la siguiente 

\begin{eqnarray}
\label{induccionMHD}
\boxed{
\frac{\partial B_{i}}{\partial t} + \frac{\partial}{\partial x_{j}}(v_{i}B_{j}-v_{j}B_{i}) = \frac{\eta}{\mu_{o}}\frac{\partial^{2}B_{i}}{\partial x_{j}^{2}}
}.
\end{eqnarray}

%\noindent esta ecuación está acoplada con la ecuación de Euler para dar la evolución de $v$ además actuando como la fuerza externa que da el comportamiento particular de los plasmas


\noindent Ahora se encuentra la ecuación de momento del fluido magnetizado, para esto se estudia el cambio del momento en el tiempo para tal fin se considera la ecuación de continuidad \eqref{continuidad} y la ecuación de Navier Stokes \eqref{NS}. Se escriben de manera conveniente tal y como sigue 

\begin{eqnarray}
\label{contindicial}
\frac{\partial \rho}{\partial t}&=&-\frac{\partial}{\partial x_{k}}(\rho u_{k})\\
\label{NSindicial}
\frac{\partial u_{i}}{\partial t} &=& -u_{k}\frac{\partial u_{i}}{\partial x_{k}} - \frac{1}{\rho}\frac{\partial p}{\partial x_{i}} + \frac{1}{\rho}\frac{\partial \sigma_{ik}}{\partial x_{k}}
\end{eqnarray}



\noindent se adiciona al problema la fuerza de lorentz, de esta manera se incluye la interacción electromagnética. En primer lugar vemos que la variación de momento del fluido está dada por la siguiente expresión 

\begin{eqnarray}
\frac{\partial}{\partial t}(\rho u_{i}) = u_{i}\frac{\partial 
\rho}{\partial t} + \rho \frac{\partial u_{i}}{\partial t} + (\vec{F}_{EM})_{i} 
\end{eqnarray}

\noindent teniendo en cuenta \eqref{contindicial} y \eqref{NSindicial}

\begin{eqnarray}
\frac{\partial}{\partial t}(\rho u_{i}) = u_{i}\left[-\frac{\partial}{\partial x_{k}}(\rho u_{k})\right] + \rho\left[-u_{k}\frac{\partial u_{i}}{\partial x_{k}} - \frac{1}{\rho}\frac{\partial p}{\partial x_{i}} + \frac{1}{\rho}\frac{\partial \sigma_{ik}}{\partial x_{k}}\right] + (\vec{j}\times\vec{B})_{i}\nonumber
\end{eqnarray}

\noindent ahora podemos reescribir un poco la ecuación ayudándonos de una regla de la cadena

\begin{eqnarray}
\frac{\partial}{\partial x_{k}}(\rho u_{i}u_{k}) = u_{k}\frac{\partial(\rho u_{i})}{\partial x_{k}}+u_{i}\frac{\partial(\rho u_{k})}{\partial x_{k}}\nonumber
\end{eqnarray}

\noindent obteniendo

\begin{eqnarray}
\frac{\partial}{\partial t}(\rho u_{i}) = -\frac{\partial}{\partial x_{k}}(\rho u_{i}u_{k})-\frac{\partial p}{\partial x_{k}}\delta_{ik} + \frac{\partial \sigma_{ik}}{\partial x_{k}}+ \left(\frac{1}{\mu_{o}}(\nabla\times\vec{B})\times\vec{B}\right)_{i}\nonumber\\
\label{momentos1}
\frac{\partial}{\partial t}(\rho u_{i})+\frac{\partial}{\partial x_{k}}\left[\rho u_{i}u_{k}+p\delta_{ik}\right] = \frac{\partial \sigma_{ik}}{\partial x_{k}}+\left(\frac{1}{\mu_{o}}(\nabla\times\vec{B})\times\vec{B}\right)_{i}
\end{eqnarray}

\noindent Considerando el segundo termino del lado derecho

\begin{eqnarray}
\left((\nabla\times\vec{B})\times\vec{B}\right)_{i}&=&\varepsilon_{ijk}\left(\varepsilon_{jmn}\frac{\partial B_{n}}{\partial x_{m}}\right)B_{k}\nonumber\\
&=&-\varepsilon_{ikj}\varepsilon_{jmn}\frac{\partial B_{n}}{\partial x_{m}}B_{k}\nonumber\\
&=&(\delta_{im}\delta_{kn}-\delta_{in}\delta_{km})\frac{\partial B_{n}}{\partial x_{m}}B_{k}\nonumber\\
&=&\delta_{im}\delta_{kn}\frac{\partial B_{n}}{\partial x_{m}}B_{k}-\delta_{in}\delta_{km}\frac{\partial B_{n}}{\partial x_{m}}B_{k}\nonumber\\
&=&B_{k}\frac{\partial B_{k}}{\partial x_{i}}-B_{k}\frac{\partial B_{i}}{\partial x_{k}}
\end{eqnarray}

\noindent se toman dos expresiones auxiliares para reescribir en forma de divergencia


    \begin{eqnarray}
    \frac{\partial}{\partial x_{i}}(B_{k}B_{k}) = 2B_{k}\frac{\partial B_{k}}{\partial x_{i}}\quad\longrightarrow\quad \boxed{B_{k}\frac{\partial B_{k}}{\partial x_{i}} = \frac{1}{2}\frac{\partial}{\partial x_{i}}(B_{k}B_{k})}
    \end{eqnarray}
    \begin{eqnarray}
    \frac{\partial}{\partial x_{k}}(B_{i}B_{k}) = B_{k}\frac{\partial B_{i}}{\partial x_{k}} + B_{i}\frac{\partial B_{k}}{\partial x_{k}} \quad\longrightarrow\quad \boxed{B_{k}\frac{\partial B_{i}}{\partial x_{k}}=\frac{\partial}{\partial x_{k}}(B_{i}B_{k})-B_{i}\frac{\partial B_{k}}{\partial x_{k}}}
    \end{eqnarray}

\noindent la fuerza de Lorentz se reescribe de la siguiente manera

\begin{eqnarray}
\left((\nabla\times\vec{B})\times\vec{B}\right)_{i} &=& \frac{\partial}{\partial x_{k}}(B_{i}B_{k})-B_{i}\frac{\partial B_{k}}{\partial x_{k}}-\frac{1}{2}\frac{\partial}{\partial x_{i}}(B_{k}B_{k})\nonumber 
\end{eqnarray}

\noindent como $\nabla\cdot\vec{B} = 0$ la ecuación se reduce a:

\begin{eqnarray}
\label{LorentzIndicial}
\left((\nabla\times\vec{B})\times\vec{B}\right)_{i} &=& \frac{\partial}{\partial x_{k}}(B_{i}B_{k})-\frac{1}{2}\frac{\partial}{\partial x_{i}}(B_{k}B_{k}) 
\end{eqnarray}

Se reemplaza \eqref{LorentzIndicial} en \eqref{momentos1} para obtener

\begin{eqnarray}
\label{LFMS}
\boxed{\frac{\partial(\rho u_{i})}{\partial t}+\frac{\partial}{\partial x_{k}}\left[\rho u_{i}u_{k}+\left(p+\frac{1}{2\mu_{o}}B^{2}\right)\delta_{ik}-\frac{1}{\mu_{o}}B_{i}B_{k}\right] = \frac{\partial \sigma_{ik}}{\partial x_{k}}}
\end{eqnarray}

Donde $B^{2}$ hace referencia al término de presión magnética y el tensor $\vec{B}\vec{B}$ es la tensión magnética. 

\section{Ecuaciones del sistema}

\noindent Las ecuaciones fundamentales en MHD empleadas se resumen de la siguiente manera

\begin{eqnarray}
\frac{\partial \rho}{\partial t} + \nabla\cdot(\rho\vec{v}) &=& 0\nonumber\\
\frac{\partial B_{i}}{\partial t} + \frac{\partial}{\partial x_{j}}(v_{i}B_{j}-v_{j}B_{i}) &=& \frac{\eta}{\mu_{o}}\frac{\partial^{2}B_{i}}{\partial x_{j}^{2}}\\
\frac{\partial(\rho u_{i})}{\partial t}+\frac{\partial}{\partial x_{k}}\left[\rho u_{i}u_{k}+\left(p+\frac{1}{2\mu_{o}}B^{2}\right)\delta_{ik}-\frac{1}{\mu_{o}}B_{i}B_{k}\right] &=& \frac{\partial \sigma_{ik}}{\partial x_{k}}\nonumber
\end{eqnarray}




\section{Parámetros adimensionales}

\begin{itemize}
    \item \textbf{Número de Reynolds Magnético}
    
    \noindent En términos de una velocidad típica del plasma $v_{o}$ y una longitud de escala $l_{o}$ típica del sistema y la magnitud del término de convección.

    \begin{eqnarray}
    R_{m} = \frac{l_{o}v_{o}}{\eta}
    \end{eqnarray}
    
    \noindent Es una medida de la fuerza de acoplamiento entre el flujo y el campo magnético. En el laboratorio usualmente se tiene $R_{m} << 1$ y el acoplamiento s débil, mientras que en la atmósfera solar se tiene $R_{m} >> 1$ y el acoplamiento es fuerte.

    \item
    
    \noindent \textbf{Número de Mach}
    
    \noindent Este número adimensional mide la velocidad del fluido $v_{o}$ en relación con la velocidad del sonido 
    
    \begin{eqnarray}
        c_{s} = \left(\frac{\gamma p_{o}}{\rho_{o}}\right)^{1/2}
    \end{eqnarray}
    este valor depende del medio físico en que se transporta el sonido la utilidad del número de Mach reside en que permite expresar la velocidad de un objeto tomando como referencia la velocidad del sonido, algo interesante desde el momento en que la velocidad del sonido cambia dependiendo de las condiciones del medio.
    
    \time 
    
    \noindent \textbf{Número de Mach-Alvén }
    
    \noindent Considerando la velocidad de Alfvén $v_{A} = B_{o}/(\mu \rho_{o})^{1/2}$, donde $B_{o}$ y $\rho_{o}$ son el campo magnético típico del sistema y la densidad del plasma típica. 
    
    \begin{eqnarray}
    M_{A} = \frac{v_{o}}{v_{a}}
    \end{eqnarray}
    
    Entonces este número relaciona la velocidad del fluido con la velocidad de Alfvén, velocidad a la cual se propagan las ondas magnetohidrodinámicas dentro del sistema estudiado 
    
    \item 
    
    \textbf{Parámetro $\beta$ del plasma}
    
    \noindent Dado que se encuentra que $B^{2}/2\mu_{o}$ es la presión de campo magnético. Considerando que la suma de la presión hidrostática y la presión magnética es constante. En un plasma con un gradiente de densidad la presión de campo magnético debe ser baja donde la densidad de partículas es alta, y viceversa. El decrecimiento de del campo magnético es causado por las corrientes diamagnéticas. El aporte que este efecto tiene dentro de un plasma se cuantifica utilizando la siguiente expresión 
    
    \begin{eqnarray}
    \beta = \frac{2\mu p_{o}}{B_{o}^{2}}=\frac{\text{Presión partículas}}{\text{Presión magnética}}
    \end{eqnarray}
    
    Ahora se puede considerar plasmas con un bajo $\beta$ , en los cuales se sitúa entre $10^{-3} < \beta < 10^{-6}$, en donde el efecto diamagnético entonces es muy pequeño. En cambio si $\beta$ es alto , el valor local de $B$ puede ser muy reducido por el plasma, esto es común encontrarlo en plasmas en el universo
\end{itemize}





\section{Física Solar }
